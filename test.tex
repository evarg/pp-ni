\documentclass[a4paper,twoside,onecolumn]{report}
\author{Norman Borlaug}
\title{Introduction to Tomography}
\date{\today}
\usepackage{graphicx}
\usepackage{tabularx}
\usepackage{amsmath}
\usepackage{amssymb}
\usepackage{amsfonts}
\usepackage[utf8]{inputenc}
\usepackage[T1]{fontenc}
\usepackage[polish]{babel}

 \begin{document}
    \maketitle

    \begin{abstract}
        Tomography refers to imaging by sections or sectioning, through
        the use of any kind of penetrating wave. A device used in
        \emph{tomography} is called a tomograph, while the image produced is a
        \emph{tomography}. The \emph{tomography} was invented by Sir Godfrey Hounsfield,
        the computed tomographic (CT) scanner, and thereby made an
        exceptional contribution to medicine. The method is used in
        radiology, archaeology, biology, geophysics, oceanography,
        materials science, astrophysics, quantum Information, and other
        sciences. In most cases it is based on the mathematical
        procedure called tomographic reconstruction.
    \end{abstract}

	\begin{equation}
		x^{n + 2} = \frac{y_{i}}{\sqrt[3]{2 + z_{j}}}
	\end{equation}

    In conventional medical X-ray \emph{tomography}, clinical staff make a sectional
    image through a body by moving an X-ray source and the film in opposite
    directions during the exposure. Consequently, structures in the focal plane
    appear sharper, while structures in other planes appear blurred (MeSH).
    By modifying the direction and extent of the movement, operators can select
    different focal planes which contain the structures of interest. Before the
    advent of more modern computer-assisted techniques, this technique,
    developed in the 1930s by the radiologist Alessandro Vallebona, proved
    useful in reducing the problem of superimposition of structures in
    projectional (shadow) radiography.
	\begin{figure}[h]
		\centering
		\includegraphics[width=0.5\textwidth]{picture1.jpg}
		\caption{Basic principle of tomography.}
	\end{figure}

\begin{tabularx}{\textwidth}{ || X | c | l || }
	\hline
	Name                                     & Source of data                 & Abbreviation \\
	\hline \hline
	Atom probe tomography                    & Atom probe                     & APT          \\
	\hline
	Computed Tomography Imaging Spectrometer & Visible light spectral imaging & CTIS         \\
	\hline
\end{tabularx}

\section{Description}
	In conventional medical X-ray \emph{tomography}, clinical staff make a sectional
	image through a body by moving an X-ray source and the film in opposite
	directions during the exposure. Consequently, structures in the focal plane
	appear sharper, while structures in other planes appear blurred (MeSH).
	By modifying the direction and extent of the movement, operators can select
	different focal planes which contain the structures of interest. Before the
	advent of more modern computer-assisted techniques, this technique,
	developed in the 1930s by the radiologist Alessandro Vallebona, proved
	useful in reducing the problem of superimposition of structures in
	projectional (shadow) radiography.

	In a 1953 article in the medical journal Chest, B. Pollak of the Fort
	William Sanatorium described the use of planography, another term for
	\emph{tomography} (Pollak).

\end{document}